\subsection{BnP structure}

The structure of the BnP algorithm is the same whatever the chosen branching-rule, queueing method, heuristics or parameters. It can be summarized by the following algorithm :

\begin{figure}[!ht]
	\centering
	\begin{minipage}{0.8\linewidth}
		\begin{algorithm}[H]
			\DontPrintSemicolon 
			\SetAlgoLined
			\KwIn{The items and their size $s_1,\dots,s_N$, the bin capacity $C$, an upper bound on the number of bins $B$, a precision $\epsilon$}
			\tcp{Initialization}
			Initialize an empty tree\;
			Initialize the queue with the root\;
			Initialize the column pool with an artificial column\;
			Initialize UB $\leftarrow B$\;
			Initialize LB $\leftarrow \ceil[\Big]{\sum s_i/C}$\;
			Process a root heuristic to find a better UB (see \ref{heuristic-root})\;
			\tcp{Tree exploration}
			\While{the queue is non-empty}{
				Pop the first node in the queue\;
				Proceed the node according to the branching rule set (see \ref{ryan-foster}, \ref{generic} and \ref{node-process})\;
				Find the branching variable and add the two child nodes in the queue\;
				Process a tree heuristic to tighten UB (see \ref{heuristic-tree})\;
				Update LB and UB with the solution found by the node or by the tree heuristic\;
				Cut branches that cannot improve the upper bound\;
				\If{$|UB-LB|<\epsilon$}{return the best solution associated to UB}
			}
			\caption{Branch and Price}
		\end{algorithm}
	\end{minipage}
\end{figure}
\noindent In the following, the simplest steps of the BnP are explained. The heuristics, the branching method and the node processing have a dedicated section.

\subsubsection{Column management}

The column pool contains all the columns which are created by the subproblem, whatever the node. We need to add an artificial column for the master problem to always be feasible. This artificial column contains all the items and have a very high cost. If the solution of a master problem contains this artificial column, we know that the master problem is infeasible because of the branching rules. Before each node processing, we have to create a local node pool containing only the patterns satisfying the branching rules of the node in order to create a solution satisfying the current branching rules. To save speed and memory space, the column pool is global to all nodes and each node has its proper node pool which is basically a filter on the global column pool according to the node branching rules. When a new pattern is created in a subproblem, it is added both in the node pool and in the global column pool. 

\subsubsection{Initial bounds}

We could have set the initial bound at $UB = + \infty$ and $LB = -\infty$ but better initial bound can be found. Indeed, we can assume that the total number of bin used is at most the number of items. Indeed, if at least one of the items cannot be packed in any bin, the problem is infeasible. Thus, we can initially set $UB = N$. The integer relaxation of \eqref{BPP-classic} gives also a lower bound for the integer \eqref{BPP-classic} : we can initially set $LB = \ceil[\Big]{\sum_i s_i/C}$ \cite{sadykov2013bin}. In addition, a root-heuristic can be proceeded to obtain a better initial UB (see \ref{heuristic-root}).

\subsubsection{Queuing method}

In the while loop, we always choose the first node in the queue but the nodes are not always  stored in the same order in the queue. When adding the two child nodes to the queue, three methods can be used. The first one puts new nodes at the beginning of the queue (LIFO) in order to proceed a Deep-First-Search in the tree. The second method puts the new nodes at the end of the queue (FIFO) in order to proceed a Breadth-First-Search in the tree. The LIFO method allows to find quickly good upper bounds but cuts branches deeply in the tree. The FIFO method allows to cut branches at a high level in the tree, letting less nodes to be explored, but is slower to get good upper bounds. The best exploration method depends on the structure of the problem and in general, there is no one better than the other. An hybrid exploration of the tree is also provided. First, the LIFO method is set in order to find quickly a good upper-bound. Once this upper bound is found, the queuing strategy switches to FIFO in order to explore the tree with a Breadth-First-Search. This method aims to find quickly an upper bound and then cut branches high in the tree. If a root heuristic is computed, then the Hybrid method is just like the LIFO method.

\subsubsection{End of node processing handling}

After the node processing (see \ref{ryan-foster}, \ref{generic} and \ref{node-process}), it is possible that the solution obtained is integer. In this case, we can see if this solution can improve the current UB and if it is the case, we can update the value of UB. We also choose the best lower bound among all those of the nodes in order to update the global LB. This allows to see how close the algorithm is from the solution by bounding the optimal solution by LB and UB at each node. When a node is infeasible, the branch processing returns a solution containing the artificial column. In this case, we know that it is useless to continue the exploration in the branch so the node is pruned by infeasibility. If the node is feasible, a branching variable is selected, two new nodes are added to the tree with new constraints and the process goes on. Just before adding the new nodes, an heuristic can be processed in order to created a feasible solution using the fractional solution provided by the node (see \ref{heuristic-tree}). This feasible solution can be used to tighten the current UB.

\subsubsection{Tree pruning}

Once the bounds are updated, we take a look at the nodes left in the queue. If a node has a lower bound larger than UB, then this node can't improve the current best solution. Thus, we can stop the exploration of the branch after this node. branches are also pruned when a node is infeasible. When $|UB - LB| < \epsilon$, then the current best solution can not be improved and corresponds to the optimal solution of \eqref{BPP-classic}. If the queue becomes empty while $UB \neq LB$, this means that either the problem has no solution or that $\epsilon$ is too small regarding to numerical errors.