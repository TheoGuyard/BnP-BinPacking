Now that the core structure of the BnP has been presented, we can focus on the node processing where two methods can be used : the Ryan \& Foster and the Generic method.

\subsection{Ryan \& Foster method}
\label{ryan-foster}

The Ryan \& Foster branching rule allows to keep a single subproblem per node but it is not as easy to solve as a classical knapsack problem. 

\subsubsection{Branching rules}

When a node is solved at optimality and that two items $i$ and $j$ have been found to create a branch, we have to create one child where $w_{ij} \leq 0$ (the down branch) and one child where $w_{ij} \geq 1$ (the up branch). For the down branch, the Ryan \& Foster branching rule simply add the constraint $x_i + x_j \leq 1$ in the subproblem in order to create patterns with items $i$ and $j$ separated. For the up branch, the constraint $x_i = x_j$ is added to the subproblem in order to create patterns containing $i$ and $j$ and patterns containing neither $i$ nor $j$ (as it is impossible to create a solution only with patterns containing both $i$ and $j$). The subproblem handle itself whether to generate patterns with (resp. without) $i$ and $j$ because the reduced cost is too high if too many pattern with (resp. without) $i$ and $j$ have already been be created.

While exploring the tree, the branching rules are added sequentially to the nodes. Thus, for nodes at the bottom of the tree, the subproblems have several branching constraints. It is possible that some combinations of constraints make the subproblem infeasible. In this case, the process of the node is stopped and the branch is cut as it is impossible to create solutions for the rest of the branch. Using such method for the node processing allows to keep a unique subproblem per node and it is simple to create the local node pool as we can only loop on the global column pool and keep only the columns satisfying the current branching rules.

\subsubsection{Subproblem resolution}

The subproblem always have the structure presented in \ref{subproblem} where the node branching constraints are added. Thus, it is possible to solve this problem using a solver like Gurobi. A dynamic programming method can also be used to solve the pricing problem. However, it wasn't implemented successfully. Despite failing to implement this second method, we present it in \ref{dynamic_rf}. 