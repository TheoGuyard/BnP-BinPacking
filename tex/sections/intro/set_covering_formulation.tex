\subsection{Branch-and-Price approach }

To address the NP-hardness of the problem, we can solve this problem with a Branch-and-Price algorithm.

\subsubsection{Set covering formulation}

We choose an other formulation of the \eqref{BPP-classic}. Let $\Pc$ the set of all combinations of items of $I$ such that their cumulative size doesn't exceed the bin capacity $C$. Each element of $\Pc$ is called a \textit{pattern}. Let $P=|\Pc|$, we have
\begin{equation*}
	\Pc = \bigg \{ \x \in \{0,1\}^{|N|}, \quad \sum_{i=1}^N x_i s_i \leq C \bigg \} = \bigg \{ \x = \sum_{p=1}^P \alpha^p \bar{x}^p, \quad \sum_{p=1}^P \alpha^p = 1, \alpha \in \{0,1\}^{|P|}  \bigg \}
\end{equation*}
Here, $\bar{x}^p$ is a pattern and $\bar{x}^p_i=1$ if the item $i$ belongs to the pattern $p$. Using this Dantzig reformulation of the set $\Pc$, we can write \eqref{BPP-classic} under its \textit{set covering formulation} :
\begin{equation}
	\label{SCBPP}
		\left \{
		\begin{array}{*5{>{\displaystyle}l}}
		\min & \sum_{p=1}^P \alpha^p \\
		\st & \sum_{p=1}^P x_i^p\alpha^p = 1 & \forall \ i = 1,\dots,N \\
		& \alpha^{p} \in \{0,1\} &  \forall \ p = 1,\dots,P
		\end{array}
		\right.
	\tag{$\Sbb\Cbb\Bbb\Pbb\Pbb$}
\end{equation}
For this formulation, $x_i^p=1$ if the item $i$ is within the pattern $p$ and $\alpha^p=1$ if the pattern $p$ is used in the solution. The optimal solution is a set of patterns and each patter correspond to a bin. The constraint ensure that each item is packed in a unique pattern/bin.

One thing to notice is that $|P|$ (the number of variables) is huge and we have to know every possible patterns. At most, $|P|$ is the number of combination available with $N$ items. Even for small instances, the number of variable is not tractable as it. Furthermore, it is needed to enumerate and store all the patterns possible in order to solve the problem which could involve a huge amount of memory. 

\subsubsection{Restricted master problem}

To handle this problem, we introduce a new set $\Pc' \subset \Pc$ of cardinal $P'$ containing only a fraction of the available patterns. We can solve \eqref{SCBPP} on this restricted number of pattern, which is more tractable. This new problem is called the \textit{restricted master problem} :
\begin{equation}
	\label{RMBPP}
	\left \{
	\begin{array}{*5{>{\displaystyle}l}}
	\min & \sum_{p=1}^{P'} \alpha^p \\
	\st & \sum_{p=1}^{P'} x_i^p\alpha^p = 1 & \forall \ i = 1,\dots,N \\
	& \sum_{p=1}^{P'} \alpha^p \leq B \\
	& \alpha^{p} \in [0,1] &  \forall \ p = 1,\dots,P'
	\end{array}
	\right.
	\tag{$\Rbb\Mbb\Bbb\Pbb\Pbb$}
\end{equation}
The solution of \eqref{RMBPP} is a sub-optimal bound of \eqref{SCBPP} as some of the optimal patterns may not be included in $\Pc'$. However, if the set $\Pc'$ is well chosen and contains the optimal patterns, then the optimal solution of \eqref{RMBPP} is the same as the optimal solution of \eqref{SCBPP}. The problem is to choose the right patterns to include in $\Pc'$. As $\Pc'$ could also have a big cardinal, we rather solve a relaxation of the master problem and that is why we have that $\alpha_{p} \in [0,1] \ \forall \ p = 1,\dots,P'$. We also add a constraint on the number of pattern allowed as we know that there are only $B$ bins available. If we have some additional informations about the problem, it is possible to tighten this bound. Solving the relaxation leads to a supra-optimal solution comparing to the integer formulation of \eqref{RMBPP} (with $ \alpha^{p} \in \{0,1\}$). We will see later how to obtain an integer solution in order to have the same solution as the optimal solution of \eqref{SCBPP} but now, we focus on how to select right patterns to include in $\Pc'$.

\subsubsection{Subproblem}
\label{subproblem}

We can write an other optimization problem called the \textit{subproblem} (or \textit{pricing problem}) which aims to generate "interesting" patterns to include in $\Pc'$. If we note $\pi$ and $\sigma$ the optimal dual variables corresponding to the first and second constraint of \eqref{RMBPP}, the pricing problem takes the following form :
\begin{equation}
	\label{SPBPP}
		\left \{
		\begin{array}{*5{>{\displaystyle}l}}
		\min & 1-\sum_{i=1}^{N} \pi_i y_i - \sigma \\
		\st & \sum_{i=1}^{N} y_i s_i \leq C \\
		& y_i \in \{0,1\} &  \forall \ i = 1,\dots,N
		\end{array}
		\right.
	\tag{$\Sbb\Pbb\Bbb\Pbb\Pbb$}
\end{equation}
where $y_i=1$ if the item $i$ is included in the pattern $\y$. The solution of the pricing problem is a feasible pattern (\ie which doesn't exceed the bin capacity). The optimal cost is called the \textit{reduced cost}. We can see that usually, the cost of a pattern is $1$ (one pattern used correspond to one bin used) but here, the cost is penalized by a term containing $\pi$ and $\sigma$. This second term penalizes patterns which can not improve the set $\Pc'$ in the sense that if a pattern $\y$ has a high reduced cost, the solution of \eqref{RMBPP} with $\Pc'$ or with $\Pc' \cup \y$ is the same. If $\opt{z}_{sp}$ denotes the optimal cost and $\opt{\y}$ the associated patter of \eqref{SPBPP} for a given $\pi$ and $\sigma$, then $\opt{\y}$ can improve the solution of \eqref{RMBPP}
if and only if $\opt{z}_{sp}<0$.

One thing very important to notice is that this pricing problem is equivalent to a knapsack problem with weights $\pi$, sizes $s$ and capacity $C$. A knapsack problem is relatively easy to solve. The key mechanism of the Branch-and-Price algorithm is to solve sequentially \eqref{RMBPP} and \eqref{SPBPP}, including the new pattern provided by the subproblem, until a solution $\opt{z_{sp}} \geq 0$ is obtained. At this stage, the continuous relaxation of \eqref{RMBPP} is solved at optimality. We now focus on a method aiming to obtain integer solutions for \eqref{RMBPP} in order to have the optimal solution of \eqref{SCBPP}. This method is called branching.

\subsubsection{Branching}

\eqref{RMBPP} is a relaxation of \eqref{SCBPP} so it gives a supra-optimal solutions. In order to have the optimal integer solution, we can introduce \textit{branching rules} (new constraints) to break the fractional property of the variables. Let's consider a tree where each node correspond to a problem like \eqref{RMBPP} with its local branching rules. When a node is solved to optimality (\ie \eqref{RMBPP} is solved sequentially until $\opt{z}_{sp} \geq 0$), we can select fractional variables in the solution. Child nodes are created under the node just solved with new constraints in order to obtain solutions where the selected branching variables remains integer for the rest of the branch. The tree is explored node after node until all the nodes are explored. The key idea is  to cut non-improving branches during the explorations so as to explore only a part of the tree.

We can not branch only on the fractional $\alpha^p$. Instead, we select a pair $(i,j)$ of items such that the variable
\begin{equation*}
	w_{ij} = \sum_{p \in \Pc' | x_i^p=x_j^p=1} \alpha^p
\end{equation*}
is fractional. In one branch, we impose that $w_{ij} \geq 1$ (items $i$ and $j$ are always together) and in the other branch, we impose that $w_{ij} \leq 0$ (items $i$ and $j$ are always separated).
