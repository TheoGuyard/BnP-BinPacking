\subsection{Tree heuristics}
\label{heuristic-tree}

Dual bounds found through the tree are proven to be quite good \cite{sadykov2013bin}. Thus, an efficient primal heuristic allowing to get good primal bounds can lead to a very successful algorithm. Three different types of heuristics are implemented. The first type relies on the MIRUP property of the \eqref{BPP-classic} \cite{delorme2016bin}. The second type of heuristics are described in \cite{wascher1996heuristics} and are base on different rounding strategies, starting from the solution of the relaxed restricted master of a node. The third type of heuristics are diving heuristics described in \cite{sadykov2013bin} for the case of the Bin-Packing with Conflicts. 

\subsubsection{MIRUP-based strategy}

The MIRUP property of the \eqref{BPP-classic} is a conjecture which is still open. It has been conjectured \cite{scheithauer1995modified} that given the solution $\opt{z}_{\Cbb\Rbb}$ the solution of the continuous relaxation of \eqref{SCBPP}, the following inequality holds :
\begin{equation}
	\label{mirup}
	\opt{z} \leq \ceil{\opt{z}_{\Cbb\Rbb}} + 1
\end{equation}
where $\opt{z}$ is the solution of \eqref{SCBPP} with the integrity constraint. $\ceil{\opt{z}_{\Cbb\Rbb}} + 1$ is called the MIRUP bound. The idea is first to obaint the solution of \eqref{RMBPP} at each node and compute the MIRUP bound. Then, the restricted master problem is solved with integrity constraint and with a new constraint :
\begin{equation*}
	\sum_{p \in \Pc'} \alpha^p \leq \ceil{\opt{z}_{\Cbb\Rbb}} + 1
\end{equation*}
If the solution is feasible and outperform the current best solution, then the best solution is updated. Adding this constraint allow to reduce drastically the search space for an integer solution and ensure that UB and LB will have only one unity of difference. However, the solving process can still be long as we solve an integer problem instead of a relaxed problem.


\subsubsection{Rounding strategies}

In 1995, Wäscher and Gau have presented several rounding strategies grouped in three different categories. This first group is called the Basic Pattern Approach and relies directly on the solution of the relaxed master problem. Method of this group have a name starting with \textbf{B}. The second group is called the Residual Pattern Approach and is based on the solution of the relaxed master problem where non-integer components have been rounded-down to zero. As this modification may lead to an infeasible solution, the methods of this groups aims to construct a feasible solution solving a Residual Problem (find new columns to add to the rounded-down solution to create a feasible solution).  Method of this group have a name starting with \textbf{R}. Finally, the third group is called the Composite Approach and is a mix of the two first groups.  Method of this group have a name starting with \textbf{C}.

Several method are proposed in \cite{wascher1996heuristics} and methods \textbf{RSUC} and \textbf{CSTAOPT} are shown to be the most effective. Only the following methods has been implemented.

\begin{paragraph}{Procedure BRUSIM}
	The simplest procedure consists of simultaneously rounding-up any non-integer component of the \eqref{RMBPP} to one. The advantages of this procedure are obvious : it is extremely fast and immediately result in a feasible solution. However, it produces very bad primal bounds.
\end{paragraph}

\begin{paragraph}{Procedure BRURED}
	Rounding-up the non-integer components simultaneously often creates an over-packaging of some items. Then it may be feasible to remove some patterns without violating the packaging constraint. Neumann and Morlock \cite{bankhofer2000quantitative} suggest to check whether a pattern can be eliminated without causing a violation of the packaging constraints. 
\end{paragraph}

\begin{paragraph}{Procedure BOPT}
	This strategy simply solve \eqref{RMBPP} on the node pool with integrity constraints on the $\alpha^p$. This can be done by any optimal solving procedure. However, it may be very long to run this procedure as the problem to solve has integrity constrains. This method provides quite good upper bounds.
\end{paragraph}

\begin{paragraph}{Procedure BRUSUC}
	Solving directly \eqref{RMBPP} with integrity constraints can be quite costly in time. The BRUSUC address to this problem by first fixing integer variable in the solution of \eqref{RMBPP}. Then, \eqref{RMBPP} will be re-optimized (still with continuous constraints) until all the variables are integer. Before each re-optimization, the variable with the higher fractional value is fixed to one.
\end{paragraph}

\begin{paragraph}{Procedure CSTAOPT}
	A cutting pattern is a pattern containing an item which is packed more than one. The CSTAOPT procedure starts by finding a feasible solution by applying the BRUSUC procedure. Then, it removes iteratively cutting patterns until there are no cutting pattern left. It will remain a residual problem to be solved as removing cutting pattern can lead to an infeasible solution. The residual pattern is solved by an optimal method.
\end{paragraph}

\subsubsection{Diving heuristics}

\toDo{Finish}