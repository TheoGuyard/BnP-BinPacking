In this section, the main ideas of the BnP algorithm implemented to solve \eqref{BPP-classic} are presented.

\subsection{Resolution method}

To solve the Bin-Packing Problem, we can use two different branching rules. The first one was proposed by Ryan \& Foster in 1981 \cite{ryan1981rn}. This branching rule allows to keep a single subproblem at each node but doesn't preserve the knapsack structure of the pricing problem. The subproblems are rather knapsack with conflicts problems, harder to solve than the usual knapsack problem. The second branching rule which can be used is a generic branching scheme introduced by Vance in 1994 \cite{vance1994solving}. This branching rule preserves the knapsack structure of the subproblems but at each node, multiple subproblems have to be solved. For the Ryan \& Foster branching rule, the pricing problem is solved with a LP solver. A dynamic programming method is also presented at the end of this report but was not implemented successfully. For the Generic branching method, the pricing problem is solved either with a LP solver or with a dynamic programming algorithm suited for the knapsack problem. To improve the speed of the algorithm, several root-heuristics are proposed in order to obtain a better initial upper bound. Some tree-heuristics allowing to get feasible solutions throughout the tree exploration are also presented. Two different ways to add the nodes to the queue are granted : FIFO and LIFO. 