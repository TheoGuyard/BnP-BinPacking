\subsection{Node processing algorithm}
\label{node-process}

To make things simpler to understand in the implementation, we only use on type of node for both branching method. A node always contains some branching sets but in the case of the Ryan \& Foster method, it has only on branching set with no rule and a coefficient equal to $B$ as we only have one subproblem per node. A node also has a local lower bound and a list of all up and down branching previously made in its branch in order to constructs the new constraints for the Ryan \& Foster method (but also to debug the BnP when needed). In the following, $\pi$ is the dual variable associated to the constraint of \eqref{RMBPP} and $\sigma$ are the constraints associated to each branching subset (only one in the case of Ryan \& Foster method). Whatever the method set to process the node, the algorithm is always the same :
\begin{figure}[!ht]
	\centering
	\begin{minipage}[t]{0.7\linewidth}
		\begin{algorithm}[H]
			\DontPrintSemicolon 
			\tcp{Initialization}
			nodePool $\ \leftarrow \ $ \textbf{filterColumnPool}()\;
			$\pi, \sigma$, solution, value $ \leftarrow$ \textbf{solveMaster}()\;
			nodelb $\leftarrow \sum \pi_i$, nodeub $\leftarrow value$, minReducedCost $\leftarrow 0$\;
			\tcp{Master problem and subproblems are sequentially solved}
			\While{true}{
				\If{solution is integer}{
					Update global UB if necessary\;	
					Prune branches with a larger LB than the new UB\;
					\textbf{break}
				}
				\For{s in branching subsets}{
					\For{r in branching rules of r}{
						reducedCost, column $\leftarrow$ \textbf{solveSubproblem}($\pi, \sigma_s$)\;
						\uIf{reducedCost < $\infty$}{
							Update minReducedCost\;
							\If{reducedCost < 0}{
								Add column to the node pool and the column pool\;
								Update the master problem data with the new column
							}
							nodelb = nodelb + reducedCost + $\sigma_s$
						} \Else {
							Node is pruned by infeasibility, \textbf{break}
						}
					}
				}
				\If{nodelb $\simeq$ nodelb}{
					Return the solution of the master problem (node is infeasible if it contains the artificial column), \textbf{break}
				}
				$\pi, \sigma$, solution, value $ \leftarrow$ \textbf{solveMaster}()\;
			}
			
			\caption{Node process}
		\end{algorithm}
	\end{minipage}
\end{figure}

In practice, a precision error is allowed when testing nodelb $\simeq$ nodeub. for the Ryan \& Foster method, the loop on $s$ and $r$ are only passed once as the node for the Ryan \& Foster method has only one branching set with an empty branching rule.