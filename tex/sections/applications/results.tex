\subsection{Numerical results}

In this section, we test the method for different instances. We also test how each parameter acts on the BnP algorithm and which parameters are better.\\

\subsubsection{Dimension influence}
The following table shows the BnP result on Falkenauer's datasets with 60 items. $n_{expl}$ is the total number of nodes explored, $\opt{z}$ is the solution found, GAP is the last dual-gap and $t$ is the running time in seconds.

\begin{figure}[!ht]
	\centering
	\scriptsize{
	\begin{minipage}{0.47\linewidth}
		\centering
		\rowcolors{2}{gray!25}{white}
		\begin{tabular}{|ccccc|}
			\hline
			\rowcolor{gray!50}
			Dataset                      & $\opt{z}$ & GAP & $n_{expl}$ & $t$ \\
			\hline
			\texttt{Falkenauer\_t60\_00} & 20.0           & 0\% & 1               & 0.2       \\
			\texttt{Falkenauer\_t60\_01} & 20.0           & 0\% & 5               & 0.3       \\
			\texttt{Falkenauer\_t60\_02} & 20.0           & 0\% & 4               & 0.0         \\
			\texttt{Falkenauer\_t60\_03} & 20.0           & 0\% & 12              & 0.1       \\
			\texttt{Falkenauer\_t60\_04} & 20.0           & 0\% & 2               & 0.0         \\
			\texttt{Falkenauer\_t60\_05} & 20.0           & 0\% & 1               & 0.0         \\
			\texttt{Falkenauer\_t60\_06} & 20.0           & 0\% & 3               & 0.1       \\
			\texttt{Falkenauer\_t60\_07} & 20.0           & 0\% & 1               & 0.0         \\
			\texttt{Falkenauer\_t60\_08} & 20.0           & 0\% & 2               & 0.0         \\
			\texttt{Falkenauer\_t60\_09} & 20.0           & 0\% & 1               & 0.1       \\
			\texttt{Falkenauer\_t60\_10} & 20.0           & 0\% & 27              & 0.2       \\
			\texttt{Falkenauer\_t60\_11} & 20.0           & 0\% & 1               & 0.0         \\
			\texttt{Falkenauer\_t60\_12}& 20.0           & 0\% & 12              & 0.1       \\
			\texttt{Falkenauer\_t60\_13} & 20.0           & 0\% & 8               & 0.0         \\
			\texttt{Falkenauer\_t60\_14} & 23.0           & 0\% & 1               & 0.9      \\
			\texttt{Falkenauer\_t60\_15} & 20.0           & 0\% & 3               & 0.0         \\
			\texttt{Falkenauer\_t60\_16} & 20.0           & 0\% & 2               & 0.0         \\
			\texttt{Falkenauer\_t60\_17} & 20.0           & 0\% & 1               & 0.1       \\
			\texttt{Falkenauer\_t60\_18} & 20.0           & 0\% & 3               & 0.0         \\
			\texttt{Falkenauer\_t60\_19} & 20.0           & 0\% & 1               & 0.0 \\
			\hline  
		\end{tabular}
	\end{minipage}
	\begin{minipage}{0.47\linewidth}
		\centering
		\rowcolors{2}{gray!25}{white}
		\begin{tabular}{|ccccc|}
			\hline
			\rowcolor{gray!50}
			Dataset                       & $\opt{z}$ & GAP  & $n_{expl}$ & $t$ \\
			\hline
			\texttt{Falkenauer\_t120\_00} & 42.0           & 4.76 & 284             & 60      \\
			\texttt{Falkenauer\_t120\_01} & 42.0           & 4.76 & 312             & 60       \\
			\texttt{Falkenauer\_t120\_02} & 42.0           & 4.76 & 283             & 60      \\
			\texttt{Falkenauer\_t120\_03} & 42.0           & 4.76 & 289             & 60      \\
			\texttt{Falkenauer\_t120\_04} & 42.0           & 4.76 & 342             & 60      \\
			\texttt{Falkenauer\_t120\_05} & 42.0           & 4.76 & 311             & 60      \\
			\texttt{Falkenauer\_t120\_06} & 42.0           & 4.76 & 330             & 60      \\
			\texttt{Falkenauer\_t120\_07} & 43.0           & 6.98 & 328             & 60      \\
			\texttt{Falkenauer\_t120\_08} & 41.0           & 2.44 & 329             & 60      \\
			\texttt{Falkenauer\_t120\_09} & 42.0           & 4.76 & 387             & 60      \\
			\texttt{Falkenauer\_t120\_10} & 42.0           & 4.76 & 342             & 60      \\
			\texttt{Falkenauer\_t120\_11} & 42.0           & 4.76 & 277             & 60      \\
			\texttt{Falkenauer\_t120\_12} & 42.0           & 4.76 & 371             & 60      \\
			\texttt{Falkenauer\_t120\_13} & 42.0           & 4.76 & 314             & 60      \\
			\texttt{Falkenauer\_t120\_14} & 42.0           & 4.76 & 334             & 60      \\
			\texttt{Falkenauer\_t120\_15} & 42.0           & 4.76 & 354             & 60      \\
			\texttt{Falkenauer\_t120\_16} & 42.0           & 4.76 & 312             & 60      \\
			\texttt{Falkenauer\_t120\_17} & 42.0           & 4.76 & 271             & 60      \\
			\texttt{Falkenauer\_t120\_18} & 42.0           & 4.76 & 344             & 60     \\
			\texttt{Falkenauer\_t120\_19} & 42.0           & 4.76 & 322             & 60      \\
			\hline
		\end{tabular}
	\end{minipage}
	}
	\caption{BnP result on Falkenauer's datasets with 60 and 120 items using the generic branching scheme with dynamic programming, a FFD root-heuristic, a BRUSUC tree-heuristic and Hybrid queueing method and a precision of $10^{-6}$.}
\end{figure}
\toDo{Anayse}

\subsubsection{Branching rule and subproblem method comparison}

We can also compare the different branching rules and the subproblem resolution method. Two good criteria which can be used are the number of nodes explored and the total running time. $rf$ denotes the Ryan \& Foster branching rule with Gurobi to solve the subproblems, $gg$ and $gd$ denotes respectively the generic branching rule for a subproblem resolution using Gurobi and dynamic programming. 

\begin{figure}[!ht]
	\centering
	\small{
	\rowcolors{2}{gray!25}{white}
	\begin{tabular}{|ccccccc|}
		\hline
		\rowcolor{gray!50}
		Dataset & $n_{expl}^{rf}$ & $n_{expl}^{gg}$ & $n_{expl}^{gd}$ & $t^{rf}$ & $t^{gg}$ & $t^{gd}$ \\
		\hline
		\texttt{Falkenauer\_t60\_00} & 2 & 2 & 2 & 2.775 & 3.039 & \textbf{1.879} \\
		\texttt{Falkenauer\_t60\_01} & \textbf{6} & 15 & 15 & 2.588 & 2.943 & \textbf{1.069} \\
		\texttt{Falkenauer\_t60\_02} & 2 & 2 & 2 & 0.985 & 1.283 & \textbf{0.435} \\
		\texttt{Falkenauer\_t60\_03} & \textbf{11} & 17 & 17 & 2.121 & 3.465 & \textbf{0.999} \\
		\texttt{Falkenauer\_t60\_04} & 2 & 2 & 2 & 1.099 & 1.174 & \textbf{0.308} \\
		\texttt{Falkenauer\_t60\_05} & 1 & 1 & 1 & 1.088 & 1.161 & \textbf{0.305} \\
		\texttt{Falkenauer\_t60\_06} & 8 & \textbf{6} & \textbf{6} & 1.482 & 1.523 & \textbf{0.340} \\
		\texttt{Falkenauer\_t60\_07} & 2 & 2 & 2 & 1.020 & 1.212 & \textbf{0.319} \\
		\texttt{Falkenauer\_t60\_08} & 1 & 1 & 1 & 0.912 & 1.038 & \textbf{0.296} \\
		\texttt{Falkenauer\_t60\_09} & 1 & 1 & 1 & 1.972 & 1.202 & \textbf{0.303} \\
		\hline  
	\end{tabular}
	\caption{Comparison of the branching rule and the subproblem resolution method on the Falkenauer's datasets using FFD root-heuristic, BRUSUC tree-heuristic, Hybrid queueing method and a precision of $10^{-6}$.}
	}
\end{figure}
\toDo{Analyse}

\subsubsection{Root-heuristic comparison}

The following table shows the root-heuristic performance. For each dataset size, we note the number of time each method gave the best result among the three different root-heuristics.

\begin{figure}[!ht]
	\centering
	\small{
		\rowcolors{2}{gray!25}{white}
		\begin{tabular}{|ccccc|}
			\hline
			\rowcolor{gray!50}
			Datasets & Number of datasets & FFD & BFD & WFD  \\
			\hline
			\texttt{Falkenauer\_u120\_xx} & 20 & 20 & 20 & 18 \\
			\texttt{Falkenauer\_u250\_xx} & 20 & 20 & 19 & 11 \\
			\texttt{Falkenauer\_u500\_xx} & 20 & 20 & 20 & 11 \\
			\texttt{Scholl\_1} & 540 & 441 & 441 & 540  \\
			\hline  
		\end{tabular}
		\caption{Comparison of the root-heuristic on the Falkenauer's datasets}
	}
\end{figure}

\subsubsection{Tree-heuristic comparison}

\subsubsection{Queuing method comparison}