\subsection{Implementation details}

\noindent The code can be found at \href{https://github.com/TheoGuyard/BnP-BinPacking}{https://github.com/TheoGuyard/BnP-BinPacking}. It is presented in detail how to run the code. In the main directory, the file \texttt{README.md} lists most of the things details below. The directory \texttt{results} contains the results under \texttt{.csv} format when running the algorithm on one dataset (\texttt{results/single\_dataset/}) or for a benchmark on a dataset directory (\texttt{results/benchmark/}). The directory \texttt{doc/} contains the papers used during the development of the BnP. The directory \texttt{src/} contains the implementation of the algorithm.\\

\subsubsection{Datasets}
The directory \texttt{data/} contains the instances used to test the BnP. The two Bin-Packing type of datasets used come from the \href{http://or.dei.unibo.it/library/bpplib}{BPPLIB} :
\begin{itemize}
	\item \href{https://link.springer.com/article/10.1007/BF00226291}{Falkenauer dataset} : Divided into two classes of 80 instances each : the first class has uniformly distributed item sizes (‘Falkenauer U’) with N between 120 and 1000, and C = 150. The instances of the second class (‘Falkenauer T’) includes the so-called triplets, i.e., groups of three items (one large, two small) that need to be assigned to the same bin in any optimal packing, with N between 60 and 501, and C = 1000. The number of items is noted in the name of the dataset.
	\item \href{https://www.sciencedirect.com/science/article/abs/pii/S0305054896000822}{Scholl dataset} : Divided into three sets of 720, 480, and 10, respectively, uniformly distributed instances with N between 50 and 500. The capacity C is between 100 and 150 (set ‘Scholl 1’), equal to 1000 (set ‘Scholl 2’), and equal to 100 000 (set ‘Scholl 3’), respectively.
\end{itemize}
If an other dataset is used, make sure that the file is under the \texttt{.bpp} format.

\subsubsection{Code file description}
File discribed below are under the folder \texttt{src/}.
\begin{itemize}
	\item \texttt{main.jl} : Parameter initialization and main code entry
	\item \texttt{bnp.jl} : Branch-and-Price core structure
	\item \texttt{master.jl} : Restricted master problem resolution method
	\item \texttt{node.jl} : Core structure of node processing resolution
	\item \texttt{subproblem.jl} : Subproblem resolution method
	\item \texttt{knapsack.jl} : Knapsack problem methods
	\item \texttt{root\_heurisitcs.jl} : Root heuristic algorithms used before the Branch-and-Price algorithm
	\item \texttt{tree\_heurisitcs.jl} : Tree heuristic algorithms used within the Branch-and-Price algorithm
	\item \texttt{benchmarks.jl} : Tools to run benchmarks for a given dataset directory 
	\item \texttt{display.jl} : Tools to display algorithm results
	\item \texttt{data.jl} : Tools to read data from a dataset file
	\item \texttt{typedef.jl} : Type definition used in the algorithm
	\item \texttt{mip.jl} : Mixed Integer Programming formulation to solve the problem with Gurobi
\end{itemize}

\subsubsection{Running code with \texttt{main.jl}}
In this file, it is possible to specify the parameters used for the algorithm. It to possible to run the BnP on a single dataset using the function \texttt{solve\_BnP()} or to use Gurobi to solve this instance with the function \texttt{solve\_MIP()}. It is also possible to run the BnP for all the instances located in the specified \texttt{benchmarksDirectory} with less than \texttt{maxItems} in it with a limit on the solving time using \texttt{maxTime} parameter. The parameters allowed are :
\begin{itemize}
	\item \texttt{branching\_rule} : \texttt{"ryan\_foster"} or \texttt{"generic"}
	\item \texttt{subproblem\_method} : \texttt{"gurobi"} or \texttt{"dynamic"}
	\item \texttt{root\_heuristic} : \texttt{"FFD"}, \texttt{"BFD"}, \texttt{"WFD"} or \texttt{"None"}
	\item \texttt{tree\_heuristic} : \texttt{"MIRUP"}, \texttt{"BRUSIM"}, \texttt{"BRURED"}, \texttt{"BOPT"}, \texttt{"BRUSUC"}, \texttt{"CSTAOPT"} or \texttt{"None"}
	\item \texttt{queueing\_method} : \texttt{"FIFO"}, \texttt{"LIFO"} or \texttt{"Hybrid"}
	\item \texttt{verbose\_level} : \texttt{1}, \texttt{2} or \texttt{3}
	\item \texttt{$\epsilon$} : between $10^{-16}$ and $10^{-4}$
	\item \texttt{maxTime} : Maximum solving time allowed in seconds
\end{itemize}
If \texttt{verbose\_level=1}, the BnP doesn't display anything. If \texttt{verbose\_level=2}, only the LB and UB found each 10 nodes are outputted. If \texttt{verbose\_level=3}, all the details of the BnP algorithm are outputted. It is not possible to use both \texttt{branching\_rule="ryan\_foster"} and \texttt{subproblem\_method="dynamic"} as only the dynamic programming algorithm for the generic branching is implemented.